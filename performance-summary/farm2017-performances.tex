%% For double-blind review submission
\documentclass[sigplan,10pt,review]{acmart}\settopmatter{printfolios=true}
%% For single-blind review submission
%\documentclass[sigplan,10pt,review]{acmart}\settopmatter{printfolios=true}
%% For final camera-ready submission
%\documentclass[sigplan,10pt]{acmart}\settopmatter{}

%% Note: Authors migrating a paper from traditional SIGPLAN
%% proceedings format to PACMPL format should change 'sigplan' to
%% 'acmsmall'.


%% Some recommended packages.
\usepackage{booktabs}   %% For formal tables:
                        %% http://ctan.org/pkg/booktabs
\usepackage{subcaption} %% For complex figures with subfigures/subcaptions
                        %% http://ctan.org/pkg/subcaption
\usepackage[utf8]{inputenc}

\makeatletter\if@ACM@journal\makeatother
%% Journal information (used by PACMPL format)
%% Supplied to authors by publisher for camera-ready submission
\acmJournal{PACMPL}
\acmVolume{1}
\acmNumber{1}
\acmArticle{1}
\acmYear{2017}
\acmMonth{1}
\acmDOI{10.1145/nnnnnnn.nnnnnnn}
\startPage{1}
\else\makeatother
%% Conference information (used by SIGPLAN proceedings format)
%% Supplied to authors by publisher for camera-ready submission
\acmConference[FARM17]{ACM SIGPLAN Conference on Functional Art, Music, Modeling and Design}{9th September, 2017}{Oxford, United Kingdom}
\acmYear{2017}
\acmISBN{978-x-xxxx-xxxx-x/YY/MM}
\acmDOI{10.1145/nnnnnnn.nnnnnnn}
\startPage{1}
\fi


%% Copyright information
%% Supplied to authors (based on authors' rights management selection;
%% see authors.acm.org) by publisher for camera-ready submission
\setcopyright{none}             %% For review submission
%\setcopyright{acmcopyright}
%\setcopyright{acmlicensed}
%\setcopyright{rightsretained}
%\copyrightyear{2017}           %% If different from \acmYear


%% Bibliography style
\bibliographystyle{ACM-Reference-Format}
%% Citation style
%% Note: author/year citations are required for papers published as an
%% issue of PACMPL.
%\citestyle{acmauthoryear}  %% For author/year citations
%\citestyle{acmnumeric}     %% For numeric citations
%\setcitestyle{nosort}      %% With 'acmnumeric', to disable automatic
                            %% sorting of references within a single citation;
                            %% e.g., \cite{Smith99,Carpenter05,Baker12}
                            %% rendered as [14,5,2] rather than [2,5,14].
%\setcitesyle{nocompress}   %% With 'acmnumeric', to disable automatic
                            %% compression of sequential references within a
                            %% single citation;
                            %% e.g., \cite{Baker12,Baker14,Baker16}
                            %% rendered as [2,3,4] rather than [2-4].



\begin{document}

%% Title information
\title[FARM 2017 Concert]{FARM 2017 Concert}         %% [Short Title] is optional;
                                        %% when present, will be used in
                                        %% header instead of Full Title.
\subtitle{Evening of Algorithmic Arts}  %% \subtitle is optional


%% Author information
%% Contents and number of authors suppressed with 'anonymous'.
%% Each author should be introduced by \author, followed by
%% \authornote (optional), \orcid (optional), \affiliation, and
%% \email.
%% An author may have multiple affiliations and/or emails; repeat the
%% appropriate command.
%% Many elements are not rendered, but should be provided for metadata
%% extraction tools.

%% Author with single affiliation.
\author{Alex McLean}
\orcid{nnnn-nnnn-nnnn-nnnn}             %% \orcid is optional
\affiliation{
%  \position{Position1}
  \department{Research Institute}              %% \department is recommended
  \institution{Deutsches Museum}            %% \institution is required
%  \streetaddress{Street1 Address1}
  \city{Munich}
%  \state{State1}
%  \postcode{Post-Code1}
  \country{Germany}
}
\email{alex@slab.org}          %% \email is recommended

%% Abstract
%% Note: \begin{abstract}...\end{abstract} environment must come
%% before \maketitle command
\begin{abstract}
A concert of performances in the FARM workshop tradition, taking place
9th September 2017 in the Old Fire Station, Oxford. The performers
will all use functional programming and related techniques, to create
live music and visuals.
\end{abstract}


%% 2012 ACM Computing Classification System (CSS) concepts
%% Generate at 'http://dl.acm.org/ccs/ccs.cfm'.
\begin{CCSXML}
<ccs2012>
<concept>
<concept_id>10010405.10010469.10010471</concept_id>
<concept_desc>Applied computing~Performing arts</concept_desc>
<concept_significance>500</concept_significance>
</concept>
</ccs2012>
\end{CCSXML}

\ccsdesc[500]{Applied computing~Performing arts}
%% End of generated code


%% Keywords
%% comma separated list
\keywords{functional programming, music, art, live video, live coding, generative art}  %% \keywords is optional

\maketitle

\section{Concert overview}

This is now the fourth year that the FARM workshop has hosted a
performance evening. Every year, after a whole day introducing and
discussing the latest work in Functional Art, Music, Modeling and
Design, we find ourselves in a FARM concert, experiencing that work in
live performance. 

We can now comfortably call this a tradition, and are extremely
grateful to ICFP for supporting it. Their financial and organisational
support allows us to maintain this as a free event, open to all
ICFP delegates (including those from our sibling workshops), as well
as members of the public. 

This year the concert will take place at the Old Fire Station in
Oxford, a unique venue with excellent facilities and
reputation. Thanks to an excellent set of proposals, we have been able
to put together an interesting and diverse range of performances, with
invited keynote performance by Alexandra C\'{a}rdenas.

\section{Performance notes}

Here we outline the evening in the form of performance notes, with 
artist biographies.

\subsection{Keynote performance: Alexandra C\'ardenas}

Alexandra C\'{a}rdenas will perform through live coding, combining her
interests in improvisation, composition, programming, live electronics
and traditional music. Alexandra projects her screen for the audience
to witness what she is writing on her computer. Using SuperDirt (a
SuperCollider implementation of the Dirt sampler for the TidalCycles
programming language) Alexandra creates her own sounds in
SuperCollider and sequences them using patterns written in real time
with the software TidalCycles.

\paragraph{Biography} Composer, programmer and improviser of music, C\'{a}rdenas has
followed a path from Western classical composition to improvisation
and live electronics. Using open source software, her work is focused
on the exploration of the musicality of code and the algorithmic
behaviour of music, especially through live coding. Currently she
lives in Berlin, Germany recently completing her masters in Sound
Studies at the Berlin University of the Arts.

\subsection{Joe Beedles}

Joe Beedles has custom-built a 
system (designed in Max) which allows for live audiovisual performance
through employing functional programming techniques, algorithms,
procedural drawing and seeded autonomous computation. In its current
state he interacts with the system in an improvisatory manner, creating
rhythmical patterns, seeding upcoming changes to the structure and
directing the flow of the performance. A direct visual component,
consisting of abstract geometry and shaded hues, works alongside the
audio to create a pseudo-synesthetic experience. There is an emphasis
placed on the live, real-time, generative aspect of the performance
with a focus on interactivity between performer, computer, audience
and the space. MSP users such as Autechre have been heavily
influential both aesthetically and conceptually to his practice -- he
likes to abstract elements of techno, noise and glitch fusing FM
synthesis alongside found sound and acoustic recordings. Visually he is
inspired by a combination of early Windows screensavers (Beziers),
building architecture (Antoni Gaudi), oscilloscopes (Robin Fox),
Op/Kinetic art (Naum Gabo) and emergent patterns through natural light
(shadows and clouds).

\paragraph{Biography}
In his work, young Manchester based audiovisual artist Joe Beedles,
draws on live recordings and synthesized noise; he focuses on concepts
surrounding club music abstraction, blurring the line between ‘the
real’ and ‘the simulated’. His current emphasis is on generative
systems for live performance, providing audiences with highly-detailed
compositions, emphasising magnified yet obscured soundscapes.  Upon
premiering his work at The Banff Centre in March 2016, Beedles has
continued to develop his live set both technically and conceptually,
culminating in solo performances, particularly within the Algorave
movement and Test Card (Manchester). Beedles – who also goes by the
artist name Native – releases on London based label Laura Lies In, his
output is varied with stylistic themes of vocal manipulation,
shimmering harmonic structures, swathes of textural ambience and
deconstructed technoise. Visually simple geometry is abstracted in
reaction to the audio in a pseudo-synesthetic fashion. His EP
‘Polaris’ was released in 2016.

\subsection{Filippo Guida}

Many studies show how synaesthetic phenomenon induced by images are
able to excite the auditory cortex, giving an evidence of the
multi-modality of sound experience itself \citep{Riddoch12} and raising
new questions regarding the complementarity of music and visual art
already suggested by John \citet{Whitney80}.

In his performance, Filippo Guida deepens some aspects of this theory, already
subject of his past works, by manipulating complex video materials (in
this case human body gestures) using sound based generative techniques
and languages:

\begin{itemize}
\item The generative process (code) projected near the video contents is a guarantee that the development of graphics materials is made using musical techniques (intrinsic feature of sound based languages).
\item The correspondences between music and body movement, already proved by multi-sensory perception studies, allow to address a gesture and any image movement as a musical relevant event, rather than abstract video materials\citep{Haga08}.

  The result is a sort of artistic application of VJ techniques through live coding languages (TidalCycles) using a custom made software to manage the video samples (VideoDirt).
\end{itemize}

\paragraph{Filippo Guido} Musician, Media Artist and SW developer. Every Filippo Guida’s works
is born from a deep interest in the mechanism of mind, irrespective of
the method used to reach the knowledge itself.  He draws his
inspiration from cognitive sciences, social studies or even philosophy
of mind using such ideas as a baseline to develop pieces of art or
music works.  The use of new technologies to create and manipulate
virtual and physical materials allow him to easily address the
creative process from an analytical point of view: an attempt doomed
to failure.  In this respect, the artwork became an evidence of the
inconsistency between consciousness and reason where the programming
language represent a tool to highlight those evidences.


\subsection{Claude Heiland-Allen -- GULCII (Graphical Untyped Lambda Calculus Interactive Interpreter)}

GULCII is an untyped lambda calculus interpreter supporting interactive
modification of a running program with graphical display of graph reduction.
Lambda calculus is a minimal prototypical functional programming language
developed by Alonzo Church in the 1930s.  Church encoding uses folds to
represent data as higher-order functions.  Dana Scott's encoding composes
algebraic data types as functions.  Each has strengths and weaknesses.

The performance is a code recital, with the internal state of the interpreter
visualized and sonified.  Part 1 introduces Church encoding.  Part 2 develops
Scott encoding.  An interlude takes in two non-terminating loops, each with
their own intrinsic computational rhythm.  Finally, Part 3 tests an equivalence
between Church and Scott numerals.

\paragraph{Biography}

Claude Heiland-Allen is an artist from London UK interested in the complex
emergent behaviour of simple systems, esoteric geometries, and mathematical
aesthetics.  Using computer software, and programming his own, has been a
part of his practice for both sound and vision since the mid 1990s.

First exposed to functional programming in his first year at Oxford University
at the turn of the century, in the form of Haskell (the language which his
present-day GULCII is implemented in), he also spends a significant portion of
his time writing code for art's sake in other languages like C and GLSL.  Over
the last decade he has performed and presented across Europe and beyond.

\subsection{Nick Rothwell}

Nick Rothwell proposes an AV performance of new work drawing on our experience
with building Clojure-based performance frameworks for dance and
digital media (see video links enclosed). This new work will draw on
aspects of earlier performance projects including functional
generation of visual geometry, algorithmic sequencing, and integration
of live coding environments into a commercial DAW for sound synthesis
and musical performance. We are especially interested in applying live
coding techniques to Karsten Schmidt's Clojure-based OpenGL software
stack, and integrating Clojure into the websocket-based Max-for-Live
performance environment Gibberwocky.


\paragraph{Biography} Nick Rothwell is a software architect, programmer and sound
designer. He was a Research Fellow on the Edinburgh Standard ML
functional programming project, and later a Web/CMS Developer for
Guardian Media Group, and has built performance systems for projects
with Ballett Frankfurt, Vienna Volksoper and Braunarts, and has worked
at STEIM (Amsterdam), CAMAC (Paris) and ZKM (Karlsruhe).  He is
currently working on choreographic visualisation tools for Wayne
McGregor|Random Dance at the Wellcome Trust, music composition for
Shobana Jeyasingh Dance Company, and with Simeon Nelson and Rob
Godman, having designed and programmed algorithmic physics animations
for large-scale outdoor projection in Poland (Skyway Festival),
Estonia (Valgus Festival, Tallinn) and Lumiere Durham.  He has
composed soundtracks for choreographers Aydin Teker (Istanbul) and
Richard Siegal (Laban), and has performed with Laurie Booth (Dance
Umbrella, New Territories), and at the Different Skies Festival
(Arcosanti, Arizona), the ICA and the Science Museum’s Dana Centre. He
has also presented work at several Clojure conferences.

\subsection{Neil C Smith -- AMEN \${} Mother Function}

One sample, one function -- a live-coded, single function demolition of
the most ubiquitous sample in modern music. This new performance work
is both an essay in conceptual minimalism and an attempt at filling
the dance floor with the aid of one wavetable and a little maths.

Entirely created within the context of a single, pure function running
at audio rate, this performance starts with a sawtooth LFO reading
from a wavetable filled with the Amen break. Through the course of the
performance, the Amen break is gradually unmade into new rhythms,
melodic and synthetic sounds. Working with a single sample at a time,
eschewing recursion and randomness, provides a restrictive but
rewarding creative challenge.

The silent \${} in the title (also used in the code), is in tribute to
Gregory C. Coleman, the original drummer of the Amen break, who died
homeless and broke.

\paragraph{Biography}

Neil C Smith‘s work explores ways of opening up and challenging the
nature of creativity itself, in particular through the use of
technology. Working sonically and visually, solo and in collaboration,
he creates live improvised performances, and context-specific
generative and interactive installations. His work exists in real-time
and real-space – each moment unique and unrepeatable. Based in Oxford,
he has shown work and performed across the UK and in Europe/Canada. He
is also lead developer of Praxis LIVE, the open-source hybrid visual
environment for live creative coding used in this
performance.


\subsection{Joseph Wilk -- :heart: :skull-crossbones:}

A musical and visual performance binding Unity, Emacs and
Supercollider with Sonic Pi. Combining the quality and performance of
the Unity Games engine, the power of manipulating text through Emacs
and the musical range and diversity of the SuperCollider engine and
Sonic Pi. While samples and software+hardware synths are the main
musical tools, a lot has gone into their creation. Samples have been
sufficiently smashed into tiny pieces through Clojure based DSP
techniques, software synths have been grown in vats of QuickCheck
(exploding with state spaces) and hardware synths have been throughly
corrupted through state mutating code (sorry). What has been done to
Emacs is probably best left to your imagination and the source control
history.

A performance which breaks down the tools and languages of
programming, to rediscover how they enable us all.

\paragraph{Biography}
Joseph Wilk's path into creating music started on a journey to use Artificial
intelligence to generate music to send his new born daughter to
sleep. It lead down a path of exploring machine creativity and then
performing live coding of music and visuals. Most of his work now
explores that intersection, creating a visual and musical curated live
experience for the viewers. Wilk likes to tell a story, and play with
concepts of programming using them in novel and unintended
ways. Exploring what code means as a musical instrument and learning
for himself, how he can express himself while sharing with people, what's
possible with code. That programming is not just a path to creating a
Silicon Valley startup but a form of self expression.

\bibliography{farm2017-performances}

\end{document}
